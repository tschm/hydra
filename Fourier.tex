%% Based on a TeXnicCenter-Template by Gyorgy SZEIDL.
%%%%%%%%%%%%%%%%%%%%%%%%%%%%%%%%%%%%%%%%%%%%%%%%%%%%%%%%%%%%%

%------------------------------------------------------------
%
\documentclass{article}
%\documentclass[AIF,Unicode,manuscript]{cedram}

% For instance the command
%          \documentclass[a4paper,12pt,reqno]{amsart}
% ensures that the paper size is a4, fonts are typeset at the size 12p
% and the equation numbers are on the right side
%
\usepackage{amsmath}%
\usepackage{amsfonts}%
\usepackage{amssymb}%
\usepackage{graphicx}

\usepackage[T1]{fontenc}
%\usepackage{beton}
%\usepackage{euler}
\usepackage{latexsym}
\usepackage{amsthm}
\usepackage{hyperref}

%% Insert here your own symbols, as the following ones:
%\newcommand{\bbR}{\mathbb{R}}
%\newcommand{\bbC}{\mathbb{C}}
%\newcommand{\bbZ}{\mathbb{Z}}


%% Repeat the preceding commands for additional authors, commenting out lines
%% which should not appear


%------------------------------------------------------------
% Theorem like environments
%

\newcommand{\C}{{\mathbb  C}}
\newcommand{\R}{{\mathbb  R}}
\newcommand{\Z}{{\mathbb  Z}}
\newcommand{\N}{{\mathbb  N}}
\newcommand{\Q}{{\mathbb  Q}}

\makeatletter
\renewcommand{\leq}{\leqslant}
\renewcommand{\geq}{\geqslant}
\renewcommand{\Re}{{\operator@font Re\,}}
\renewcommand{\Im}{{\operator@font Im\,}}
\makeatother

\def\abs#1{\left|#1\right|}

\newcommand{\footremember}[2]{%
    \footnote{#2}
    \newcounter{#1}
    \setcounter{#1}{\value{footnote}}%
}
\newcommand{\footrecall}[1]{%
    \footnotemark[\value{#1}]%
}
%--------------------------------------------------------
\begin{document}

\title{Taming a Hydra of Singularities in Fourier Space}

\author{
    Thomas Schmelzer\footremember{Stanford}{Stanford University, USA}\footnote{ADIA, Abu Dhabi, UAE} \footnote{\href{mailto:thomas.schmelzer@gmail.com}{thomas.schmelzer@gmail.com}}%
    \and Emmanuel J. Candes\footrecall{Stanford} \footnote{\href{mailto:candas@stanfod.edu}{candes@stanford.edu}}%
}

%\author{Thomas Schmelzer and Emmanuel J. Candes}
%\address[]
%{Thomas Schmelzer \newline
%\indent Stanford University, USA \newline
%\indent ADIA, Abu Dhabi, UAE
%}
%\email[]{thomas.schmelzer@gmail.com}

%\author{Folkmar Bornemann}
%\curraddr[A.~Two]{TU Munich\newline%
%\indent Author Two current address, line 2}%
%\email[A.~Two]{author-two@authortwo-inst.hu}%
%\urladdr{http://www.authortwo.uni-atwo.hu}
%
%\author{Emmanuel J. Candes}
%\curraddr[]{Stanford University, USA}
%\email[]{candes@stanford.edu}

%\urladdr{http://www.authortwo.uni-atwo.hu}

%
%\thanks{Thanks for Author One.}
%\thanks{Thanks for Author Two.}
%\thanks{This paper is in final form and no version of it will be submitted for
%publication elsewhere.}
\date{\today}
%\subjclass{Primary 05C38, 15A15; Secondary 05A15, 15A18} %
%\keywords{Keyword one, keyword two etc.}%
%\dedicatory{Dedicated to Giuliana Bordigoni}

%\begin{abstract}
%This is a sample document which shows the most important features of the AMS Journal
%Article class.
%\end{abstract}
\maketitle


\section{Introduction}
In \cite{BornemannSchmelzer} Bornemann and Schmelzer invited the reader to contemplate
a remarkable limit problem which involves an oscillatory integral of extreme nature.
Consider a function $f: \R \to \R$ that is integrable\footnote{A function $f: \R \to \R$ is integrable if $\int_{-\infty}^\infty f(x) dx$ exists. Bornemann and Schmelzer solved the problem for a larger class of functions. In their analysis is was sufficient for $f$ to be bounded and continuous. }.
Does the sequence of violently oscillatory integrals
\begin{equation}\label{eq1}
I_n[f] = \frac{1}{\pi}\int_0^\pi f(\tan^{[n]}x)\,dx, \qquad n=1,2,3,\ldots,
\end{equation}
have a limit as $n$ approaches infinity? If so, what is the limit?
Here $\tan^{[n]}x = \tan \circ \tan \circ \cdots \circ \tan x$ denotes the n-fold iteration of the $\tan$ function.

It is the dramatic behavior of $\tan^{[n]}x$ that inspired Bornemann to call this function the Hydra of singularities. The aggressive nature of this function is revealed already for small $n$. As $x$ moves from $0$ to $\pi$, the values of $\tan x$ go all the way from $0$ to $+\infty$ and then from $-\infty$ to $0$ after passing the singularity at $\pi/2$. The singularity at $\pi/2$ breeds infinitely many new singularities located at $x_k$ where $\tan x_k = (k+1/2) \pi$ with $k \in \Z$. Note that the values $x_k$ accumulate at $\pi/2$.

This process repeats for each further iteration of $\tan$. Each singularity breeds countably many new singularities which accumulate in their respective ancestor. In Figure~1 we have tried to illustrate this rather wild behavior of the $\tan^{[n]}x$ iteration.

\begin{figure}[hb]
\begin{center}
\hspace*{-0.5mm}\includegraphics[scale=0.668]{img/tan2}\;\includegraphics[scale=0.668]{img/tan3}
\end{center}
\caption{Graph of  $\tan^{[2]}(x)$ (left) and $\tan^{[3]}(x)$ (right).}
\end{figure}

This paper is a brief companion for the original paper by Bornemann and Schmelzer. The approach taken here could probably be extended for a larger class of functions. This would require a careful analysis shading light away from the central ideas.
However, Bornemann and Schmelzer \cite{BornemannSchmelzer} gave already an elementary proof for a large class of functions and therefore our focus is on a short and elegant analysis
using tools from France, in particular Fourier analysis and Cauchy integrals.

\section{Into Fourier space}
The Fourier transform of $f$ exists but does not have to be integrable. This is an additional requirement for the analysis given here.
If both $f$ and its Fourier transform
\[
\hat{f}(\xi) := \int_{-\infty}^{\infty} f(x)\ e^{- 2\pi i x \xi}\,dx,\qquad \xi \in \R
\]
are integrable then for almost every $x$ (and for all $x$ if $f$ is continuous)
$f$ can be represented as the inverse transform of $\hat f$
\[
f(x) = \int_{-\infty}^\infty \hat f(\xi) e^{2 i \pi x \xi} \, d\xi.
\]
And therefore
\[
f(\tan^{[n]}x) = \int_{-\infty}^\infty \hat f(\xi) e^{2 i \pi \xi \tan^{[n]}x} \, d\xi.
\]
We insert this term into (\ref{eq1}) and restate the problem as
\[
I_n[f] = \frac{1}{\pi}\int_0^\pi \int_{-\infty}^\infty \hat f(\xi) e^{2 i \pi \xi \tan^{[n]}x} \, d\xi \,dx.
\]
Since $\hat f$ is integrable the integral
\[
\frac{1}{\pi}\int_0^\pi \int_{-\infty}^\infty \abs{\hat f(\xi) e^{2 i \pi \xi \tan^{[n]}x}} \, d\xi \,dx
\]
exists. We can therefore apply Fubini's theorem and get
\begin{equation}\label{eq3}
I_n[f] = \int_{-\infty}^\infty \hat f(\xi) \frac{1}{\pi} \int_0^\pi  e^{2 i \pi \xi \tan^{[n]}x} \,dx \,d\xi.
\end{equation}

\section{The inner Fourier integral}

Still the problem does not look any simpler. The Hydra is lurking now in the inner integral
\begin{equation}\label{Fourier1}
\frac{1}{\pi} \int_0^\pi  e^{2 i \pi \xi \tan^{[n]}x} \,dx \qquad \xi \in \R.
\end{equation}
The $\tan$ function maps the upper halfplane into itself.
Therefore, for $\xi \geq 0$ the integrand is bounded in the upper halfplane.
We get using dominated convergence
\[
\lim_{\epsilon\downarrow 0} \frac{1}{\pi} \int_0^\pi  e^{2 i \pi \xi \tan^{[n]}(x+\epsilon\,i)} \,dx =  \frac{1}{\pi} \int_0^\pi  e^{2 i \pi \xi \tan^{[n]}x} \,dx \qquad \xi \geq 0.
\]
The function  $e^{2 i \pi \xi \tan^{[n]}z}$ is analytic in the upper halfplane.
We can therefore apply Cauchy's theorem.
As a contour we choose the rectangle with corners $(0,\epsilon\,i),(\pi,\epsilon\,i),(\pi,a\,i),(0,a\,i)$.
The contributions from both vertical vertexes vanish as $\tan$ is periodic. And therefore
\[
\lim_{\epsilon\downarrow 0}  \frac{1}{\pi} \int_0^\pi  e^{2 i \pi \xi \tan^{[n]}(x+\epsilon\,i)} \,dx = \frac{1}{\pi} \int_0^\pi  e^{2 i \pi \xi \tan^{[n]}(x+a\,i)} \,dx \qquad \xi \geq 0, a > 0.
\]
Hence we can integrate on any parallel line above the real line.
In the extreme case we can move $a$ towards infinity and as
\[
\lim_{a \uparrow \infty} \tan^{[n]}(x+a\,i)= i \lim_{a \uparrow \infty} \tanh^{[n]} (a - i\,x)= i \tanh^{[n-1]}1
\]
we get
\[
\frac{1}{\pi} \int_0^\pi  e^{2 i \pi \tan^{[n]}x \xi} \,dx = e^{-2 \pi \xi \tanh^{[n-1]}1} \qquad \xi \geq 0.
\]
For $\xi \leq 0$ the same argument applied in the lower halfplane yields
\[
\frac{1}{\pi} \int_0^\pi  e^{2 i \pi \tan^{[n]}x \xi} \,dx = e^{2 \pi \xi \tanh^{[n-1]}1} \qquad \xi \leq 0.
\]
For $n \to \infty$ both integrals converge to $1$ and hence
\[
\lim_{n \uparrow \infty} I_n[f] = \int_{-\infty}^\infty \hat f(\xi)\,d\xi = f(0).
\]
In the last step we have applied Parseval's theorem.

\section{From Fourier to Hardy}

Comparing Equation (\ref{Fourier1}) and Equation (\ref{eq1}) reveals that the inner Fourier integral is just a special case of (\ref{eq1}) with $f(x) = e^{2 i \pi \xi x}$.
It may seem that we have made use of special properties of this particular function $f(x)$, but in fact the results generalises for a larger class of functions $f$.
To transfer our arguments from above we need to assume that $f(z)$ is analytic and bounded in the upper halfplane.
However, the space of bounded analytic functions in the upper halfplane is the Hardy space $H^{\infty}$.
So, let $F \in H^\infty$, then $\|F\|_{H^\infty} = \sup_{\Im z > 0}|F(z)| < \infty$. The function $f$ may be interpreted as the non-tangential limit of $F$, that is $f(x) = F(x+iy)$ for $y \downarrow 0$. This implies $f$ is bounded, too.


Assuming that $f$ is continuous and bounded on the real line this resembles a kind of boundary value problem. However, the analytic extension of $f$ is rarely bounded in the upper halfplane.

Now, from potential theory (see \cite[Thms. 15.1a, 15.4d]{Hen}) we know that there is a function $F(z)$, holomorphic
in the upper complex half plane $\Im z >0$, such that the harmonic function $\Re F(z)$ is bounded and has boundary values given by $f$, that is,
\begin{equation}\label{eq.F}
\Re F(x+ i y) \to f(x), \qquad x\in\R,
\end{equation}
as the real number $y$ approaches zero from above. The holomorphic function $F$ is \emph{unique}
up to a purely imaginary additive constant. For the sake of simplicity of our presentation, we will further
{\em assume that $F$ itself, not just $\Re F$, is bounded}\/; this additional assumption is dropped
in the elementary, real analysis proof given in \cite{BornemannSchmelzer}.

Therefore
\[
\frac{1}{\pi}\int_0^\pi f(\tan^{[n]}x)\,dx = \Re \frac{1}{\pi} \int_0^\pi F(\tan^{[n]}x)\,dx = \Re F(i \tanh^{[n-1]}1).
\]
Taking the limit for $n \to \infty$ reveals:
\[
\lim_{n \uparrow \infty} \frac{1}{\pi}\int_0^\pi f(\tan^{[n]}x)\,dx = f(0).
\]
Arguably, there are simpler ways to evaluate $f$ at $0$.
However, they all lack the drama, brutality and beauty of the Hydra.


\bibliographystyle{amsplain}
\bibliography{bib/Hydra}
\end{document}
